\documentclass{article}
\usepackage[utf8]{inputenc}
\usepackage[spanish]{babel}
\usepackage[T1]{fontenc}
\usepackage{booktabs}
\usepackage{array}
\usepackage{graphicx}
\usepackage{float}
\usepackage{geometry}
\usepackage{enumitem}


\geometry{margin=1.3in, headheight=30pt}
\setlist[itemize]{noitemsep}


\setlength{\parskip}{1em}  % Ajusta la distancia entre párrafos


\title{Resultados Encuesta sobre Programación de Actividades en el Área de Mantenimiento}
\author{Emilio Bravo Maturana}
\date{\today}

\begin{document}
\maketitle

\section*{Resumen Ejecutivo}

Se llevó a cabo una encuesta a programadores del área de mantenimiento en las operaciones Los Bronces y El Soldado de Angloamerican, con el objetivo de identificar los principales desafíos y oportunidades de mejora en el proceso de programación.

Los resultados de la encuesta indican que la programación de las operaciones mineras consume una cantidad significativa de tiempo y recursos del equipo responsable. Más de la mitad de los encuestados dedica entre 20 y 40 horas semanales a estas tareas, reflejando el carácter demandante del proceso. Además, las respuestas evidencian una alta heterogeneidad en las funciones y responsabilidades, así como una marcada variabilidad en las condiciones operativas, con cambios de último minuto que ocurren de forma frecuente (al menos semanalmente en el 90\% de los casos).

Entre los principales problemas identificados destacan las dificultades relacionadas con la disponibilidad de materiales y recursos, la falta de información integrada y las limitaciones en la coordinación entre distintas áreas involucradas. Estos hallazgos sugieren que la automatización del proceso de programación, junto con la integración de datos en tiempo real y la mejora en la comunicación interáreas, podría contribuir a reducir la carga horaria, minimizar los errores más comunes y mejorar la capacidad de respuesta ante cambios imprevistos.


\section*{Metodología}

La encuesta se aplicó a un total de 19 programadores de las operaciones Los Bronces y El Soldado de Anglo American, mediante una plataforma en línea durante el mes de diciembre de 2024. Se trató de un cuestionario breve, compuesto por 7 preguntas. La mediana del tiempo de respuesta fue de aproximadamente 5 minutos y 30 segundos. 

Las preguntas estuvieron diseñadas para estimar el tiempo dedicado a la programación, la distribución de dicho tiempo entre distintas actividades, la frecuencia e impacto de errores, y las condiciones cambiantes que inciden en la planificación.

A continuación se presentan los resultados y el análisis de las respuestas obtenidas.

\section*{Resultados de la Encuesta}

En esta sección se presentan los principales hallazgos de la encuesta, organizados en función de temas clave. Primero se describe la carga de trabajo y la asignación de recursos en el proceso de programación, luego se aborda la frecuencia de cambios de último minuto, y finalmente se analizan los errores más frecuentes asociados a la planificación manual.

\subsection*{Carga de Trabajo y Recursos}

La cantidad de horas dedicadas a la programación, la distribución de dichas horas entre diversas actividades, así como el porcentaje de la jornada diaria involucrada y el número de personas que participan, ofrecen una visión integral del nivel de recursos que demanda este proceso.

\subsubsection*{Pregunta 1: ¿Cuántas horas a la semana dedica el equipo de programación a generar el cronograma de tareas?}



\begin{table}[H]
    \centering
    \begin{tabular}{lcc}
        \toprule
        \textbf{Rango de horas} & \textbf{Porcentaje} & \textbf{Cantidad} \\
        \midrule
        Menos de 10 horas & 5\% & 1 \\
        Entre 10 y 20 horas & 32\% & 6 \\
        Entre 20 y 40 horas & 53\% & 10 \\
        Más de 40 horas & 11\% & 2 \\
        \bottomrule
    \end{tabular}
    \caption{Horas semanales dedicadas a la programación.}
    \label{tab:horas_semanales}
\end{table}

Es notable que más de la mitad del equipo (53\%) dedica entre 20 y 40 horas semanales a la programación. Esto sugiere que la planificación consume una parte significativa de la jornada, equiparable a medio tiempo laboral o incluso más, en algunos casos.

\subsubsection*{Pregunta 2: ¿Qué porcentaje del tiempo total de programación está dedicado a las siguientes actividades?}

\begin{table}[H]
    \centering
    \begin{tabular}{lccc c}
        \toprule
        \textbf{Actividad} & \textbf{Promedio (\%)} & \textbf{Mínimo (\%)} & \textbf{Máximo (\%)} & \textbf{Cantidad} \\
        \midrule
        Generar el borrador inicial & 34.79 & 0.00 & 66.00 & 19 \\
        Resolver conflictos de recursos & 23.79 & 10.00 & 70.00 & 19 \\
        Validar con otras áreas & 19.68 & 10.00 & 40.00 & 19 \\
        Realizar ajustes finales & 21.74 & 0.00 & 50.00 & 19 \\
        \bottomrule
    \end{tabular}
    \caption{Distribución porcentual del tiempo dedicado a diferentes actividades de programación.}
    \label{tab:distribucion_actividades}
\end{table}

Llama la atención que algunas personas reportan dedicar un 0\% del tiempo a la generación del borrador inicial o a la realización de ajustes finales. Esto podría indicar que en ciertos equipos o roles se omiten estas fases, ya sea porque el borrador inicial es proporcionado por otra área, o porque la persona encuestada se encarga únicamente de la validación o resolución de conflictos. Esta heterogeneidad sugiere la necesidad de mayor estandarización del proceso o de una herramienta automatizada que se adapte a diversos flujos de trabajo.

\subsubsection*{Pregunta 3: ¿Cuántas personas del equipo participan activamente en la programación y qué porcentaje de su jornada diaria dedican a esta actividad?}

A partir del análisis de las 19 respuestas obtenidas, se observa una gran variabilidad en el porcentaje de jornada diaria dedicado a la programación y en el número de personas involucradas. Las estadísticas resumen se presentan en la siguiente tabla:

\begin{table}[H]
    \centering
    \begin{tabular}{lcc}
        \toprule
        & \textbf{Promedio} & \textbf{Desviación Estándar} \\
        \midrule
        Número de personas & 2.79 & 1.47 \\
        Porcentaje de jornada diaria (\%) & 64.21 & 30.27 \\
        \bottomrule
    \end{tabular}
    \caption{Estadísticas resumen del porcentaje de jornada diaria y número de personas involucradas, basadas en las 19 respuestas.}
    \label{tab:estadisticas_resumen_jornada}
\end{table}

Estos resultados reflejan una dispersión significativa: en algunos casos, un equipo reducido (2 personas) dedica el 100\% de su jornada a la programación, mientras que en otros equipos, más numerosos, el porcentaje de tiempo invertido es mucho menor. Esta variabilidad sugiere la ausencia de un proceso estandarizado y la dependencia de factores operativos específicos.

\subsection*{Desafíos Operativos en la Programación}


Esta sección aborda los principales retos enfrentados durante el proceso de programación, incluyendo tanto los cambios repentinos que reflejan el dinamismo del entorno operativo como los errores recurrentes que complican la planificación. Comprender la frecuencia y las causas de estos desafíos es fundamental para identificar oportunidades de mejora, especialmente en el contexto de una posible automatización del proceso.

\subsubsection*{Pregunta 4: ¿Con qué frecuencia ocurren cambios de último minuto en las tareas o recursos disponibles?}

\begin{table}[H]
    \centering
    \begin{tabular}{lcc}
        \toprule
        \textbf{Frecuencia} & \textbf{Porcentaje} & \textbf{Cantidad} \\
        \midrule
        Varias veces por semana & 42\% & 8 \\
        Semanalmente & 47\% & 9 \\
        Pocas veces al mes & 5\% & 1 \\
        Rara vez & 5\% & 1 \\
        \bottomrule
    \end{tabular}
    \caption{Frecuencia de cambios de último minuto en las tareas o recursos.}
    \label{tab:cambios_ultimo_minuto}
\end{table}

Casi un 90\% del equipo enfrenta cambios de último minuto al menos una vez por semana. Esto evidencia un entorno altamente dinámico. Un sistema automatizado podría reaccionar más rápido ante estos cambios, reduciendo el impacto en la operación.



\subsubsection*{Pregunta 5: ¿Qué tipos de errores son más comunes durante la programación manual?}

\begin{table}[H]
    \centering
    \begin{tabular}{p{7cm}cc}
        \toprule
        \textbf{Tipo de error} & \textbf{Porcentaje} & \textbf{Cantidad} \\
        \midrule
        Herramientas o equipos no disponibles & 16\% & 3 \\
        Materiales o repuestos no disponibles & 74\% & 14 \\
        El equipo sobre el cual se programó la mantención se encuentra ocupado & 11\% & 2 \\
        \bottomrule
    \end{tabular}
    \caption{Tipos de errores más comunes durante la programación manual.}
    \label{tab:errores_comunes}
\end{table}

Destaca que el 74\% de los encuestados identifica la falta de materiales o repuestos disponibles como el error más frecuente. Esta clase de inconvenientes podría reducirse con un sistema automatizado que mantenga actualizada la información de inventarios y recursos.


\subsubsection*{Pregunta 6: ¿Qué otros errores considera que podrían ocurrir durante el proceso de programación?}

Además de las categorías previamente definidas, la encuesta incluyó una pregunta opcional para que los respondientes indicaran otros errores frecuentes que consideraran relevantes. En sus respuestas, se destacan principalmente la falta de coordinación y comunicación entre distintas áreas, así como la dificultad para alinear horarios con especialistas y asegurar la disponibilidad de recursos auxiliares. Asimismo, se mencionan con frecuencia los cambios de último minuto en los planes y la información deficiente proveniente de algunos sectores, como bodega.

En conjunto, estos comentarios cualitativos refuerzan la idea de que el proceso de planificación manual enfrenta retos que van más allá de la falta de materiales o equipos. La interacción con otras áreas, la necesidad de contar con información actualizada y la coordinación de tareas especializadas agregan complejidad y abren la puerta a la consideración de herramientas automatizadas que integren datos en tiempo real y mejoren la comunicación entre los diversos actores involucrados.

\subsubsection*{Pregunta 7: En promedio, ¿cuántos de estos errores ocurren por mes?}

\begin{table}[H]
    \centering
    \begin{tabular}{p{7cm}cccc}
        \toprule
        \textbf{Tipo de error} & \textbf{Promedio} & \textbf{Mínimo} & \textbf{Máximo} & \textbf{Cantidad} \\
        \midrule
        Cuadrilla no disponible & 3.47 & 0.00 & 10.00 & 17 \\
        Herramientas o equipos no disponibles & 3.87 & 1.00 & 12.00 & 15 \\
        Naves o talleres no disponibles & 2.69 & 0.00 & 10.00 & 13 \\
        Omisión de tareas críticas & 2.40 & 0.00 & 6.00 & 15 \\
        El equipo sobre el cual se programó la mantención se encuentra ocupado & 3.86 & 1.00 & 12.00 & 14 \\
        Materiales o repuestos no disponibles & 6.25 & 1.00 & 14.00 & 16 \\
        \bottomrule
    \end{tabular}
    \caption{Frecuencia promedio mensual de errores específicos.}
    \label{tab:errores_especificos}
\end{table}

La falta de materiales o repuestos disponibles vuelve a ser un factor crítico, con un promedio superior a 6 eventos por mes. Esto confirma la relevancia de contar con información integrada sobre la disponibilidad de recursos para mejorar la eficiencia del proceso de programación.
\section*{Análisis y Discusión}

Los resultados evidencian un entorno de programación altamente demandante y variable. El tiempo dedicado (en muchos casos entre 20 y 40 horas semanales) indica que la planificación es una tarea compleja que no sólo consume recursos, sino que además se ve afectada por cambios frecuentes. El alto porcentaje de actividades dedicadas a generar borradores y resolver conflictos de recursos, sumado a la gran dispersión en las cargas de trabajo, sugiere la ausencia de procesos estandarizados y herramientas unificadas.

El hallazgo más significativo es la frecuencia de errores relacionados con la disponibilidad de materiales, repuestos y equipos. Estos errores no sólo son comunes, sino que se presentan varias veces al mes, retrasando las operaciones y requiriendo ajustes constantes. La automatización de la programación, con integración a sistemas de inventarios y mantención, podría reducir significativamente estos problemas, al proporcionar información actualizada sobre disponibilidad de recursos.

Por otro lado, la presencia de algunos encuestados que no dedican tiempo a ciertas actividades (por ejemplo, 0\% a generar el borrador inicial) denota una heterogeneidad en roles y responsabilidades. Una herramienta automatizada y flexible podría estandarizar o al menos facilitar la integración entre distintos enfoques de programación, reduciendo la dependencia de procedimientos informales y mejorando la trazabilidad de las decisiones.

\section*{Conclusiones}

Los datos recopilados muestran un panorama complejo. El proceso de programación actual:

\begin{itemize}
    \item Consume una cantidad significativa de tiempo.
    \item Se ve alterado con frecuencia por cambios de último minuto.
    \item Se ve impactado por errores y dificultades relacionadas con la disponibilidad de recursos.
    \item Presenta variaciones en el rol de los programadores y la distribución de sus tareas.
\end{itemize}

Estos hallazgos apoyan la idea de que la automatización del proceso de programación podría ofrecer beneficios importantes: reducción del tiempo dedicado a tareas repetitivas, mejor capacidad de respuesta a cambios repentinos, mayor integración de datos y reducción de errores frecuentes. En consecuencia, avanzar hacia una herramienta automatizada e integrada podría representar un paso significativo para mejorar la eficiencia y la confiabilidad del proceso de programación en las operaciones mineras.

\end{document}
